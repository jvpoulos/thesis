\chapter{Using Land Lotteries for Natural Experiments on Wealth and Power} \label{land-lotteries}

The hypothesis that personal wealth influences political power has a long tradition in political theory, but has rarely been systematically studied. \citet{ferguson1767}, for instance, theorizes that unequal divisions of property produce inequalities in political power. \citet{madison1787} similarly argues that representative bodies reflect the sentiments that arise from the unequal distributions of property.\footnote{Madison does not fear the aristocracy; rather, he is more concerned about the possibility of an ``overbearing majority'' of propertyless men that would restrict property rights. For this reason, Madison argued in favor of property qualifications for voters during the Constitutional Convention, concluding, ``the freeholders of the Country would be the safest depositories of Republican liberty.''} \citet{beard1913} argues the framers of the U.S. Constitution designed the document to protect commercial property interests from the reach of popular majorities.\footnote{The specific property interests Beard refers to are ``money, public securities, manufactures, and trade and shipping,'' which were disadvantaged by the Articles of Confederation and instead favored small farmers and debtors. Beard argues the Framers were ``personally interested in, and derived economic advantages from'' the Constitution.} 

It is unclear whether wealth causes elites to seek office and vote in favor of their vested interests, or whether officeholding and the ideology of self-interest are simply correlated with unobserved personal characteristics related to the accumulation of personal wealth. For this reason, observational studies on the relationship between personal wealth or economic class and officeholding or ideology suffer from endogeneity bias. \citet{carnes2012}, for example, finds members of the U.S. House of Representatives with white-collar backgrounds vote more conservatively on economic policy than members with blue-collar backgrounds. \citet{doherty2006} estimates the effect of randomly-induced wealth on economic policy beliefs on Massachusetts lottery winners, finding lottery wealth increases opposition to estate taxes and government redistribution. \citet{rossi2014} exploits the random allocation of land in $16^{th}$ century Argentina to identify the effect of wealth on dynastic officeholding, showing that families receiving more valuable land closer to the city of Buenos Aires have a higher probability of ex-post officeholding. 

These observational studies suffer from sample selection bias due to non-random assignment of experimental groups. Chapter \ref{ga-lottery} explores the relationship between wealth and officeholding using a natural experiment in public land decentralization where land wealth was actually randomly allocated to lottery participants. The random allocation of land wealth allows us to obtain unbiased estimates of the effect of wealth on officeholding. Section \ref{land-lotteries-background} overviews the historical context of the first two Georgia land lotteries, which provide the data for the natural experiment in Chapter \ref{ga-lottery}.% It also overviews the historical background of the only other large-scale land lottery outside of Georgia, occurring in Oklahoma in 1901. Future research could replicate the natural experiment of Chapter \ref{ga-lottery} in the context of the Oklahoma lottery. 

\section{Historical Background of Land Lotteries} \label{land-lotteries-background}

In 1805 and 1807, the state of Georgia conducted the first two public land lotteries in U.S. history, only to be replicated in six subsequent Georgia land lotteries and Oklahoma's 1901 lottery. Approximately two million acres were redistributed in the Oklahoma lottery, compared to more than one million acres in the 1805 lottery, 2.2 million acres in the 1807 lottery, and more than 27 acres in all eight Georgia lotteries \citep{graham2004}. 

\subsection{1805 Georgia Land Lottery}

In 1802, the Creek Nation signed a treaty at Fort Wilkinson ceding territory south of the Oconee and Altamaha rivers. The first mechanism used for decentralizing state lands was the ``headright system'' in which heads of household were entitled to 200 acres of land and additionally 50 acres for each family member or slave (capped at 1,000 acres). The headright system favored the wealthy and well-connected, who could receive more land by obtaining the approval of the Assembly and the Governor, who at the time was appointed by the legislature \citep{meyers2012}. 

The end of the headright system and the birth of the public land lottery followed from two well known scandals. In what is referred to as the Pine Barrens scandal, three different Georgia governors signed off on grants of millions of acres of land --- more than actually existed --- to a few land speculators. The Pine Barrens scandal was followed by the Yazoo Act of 1795, in which the legislature allocated up to 50 million acres of ceded land to four speculation companies for a half-million dollars in return for bribes. Following a public anti-corruption campaign, voters installed a mostly new government in the election of 1796, which quickly nullified the Yazoo Act and grants made under the Act \citep{meyers2012}. 

Georgia's new Assembly responded with a novel system of land redistribution: a public land lottery. The Act of 11 May 1803 outlined the rules and procedures of the 1805 lottery, which served as a model for subsequent Georgia land lotteries. The lottery prevented the accumulation of land by speculators and was inherently less corruptible than the headright system. \citet{chappell1874} remarks, 

\begin{quotation}
	... [E]very temptation and means for the practice of fraud and corruption was taken away. For who was going to bribe the members of the Legislature or other public functionaries, high or low, when it was rendered utterly impossible by the very system adopted, for the corruptor to make or secure anything by means of the bribery?
\end{quotation}

The Act created three new counties using land from the former Creek territories: the territory lying south of the Oconee river was split into two counties, the northern part called Baldwin and the southern part called Wilkinson, with each county divided into five districts; Wayne county, positioned south of the Altamaha river, was divided into three districts \citep{clayton1812}.\footnote{Figure \ref{map} shows the ceded lands that were at stake in the 1805 and 1807 lotteries.}

\begin{figure}[htbp] 
	\centering
	\includegraphics[width=0.8\textwidth]{/media/jason/Dropbox/ga-lottery-local/paper-drafts/paper-sixth/county-map.png} 
	\caption{Map of Georgia with 1807 county boundaries \citep{long1995atlas}. The shaded counties are original counties created by the Acts of 11 May 1803 and 9 June 1806.\label{map}}
\end{figure}

For each district, one surveyor was appointed by the legislature to map the area into square lots and to return the results to the Surveyor General. The Act outlines the process of distributing the surveyed lots in the lottery:

\begin{quotation}
	After the surveying is completed, and the returns made to the Surveyor General, his Excellency the Governor shall cause tickets to be made out whereby all the numbers of the surveys in the different districts shall be represented, which tickets shall be put into a box to constitute prizes, with others to be denominated blanks, of which blanks the number or amount shall be determined, by subtracting the number of prizes from the whole number of draws.
\end{quotation} 

Registration for eligible citizens took place in the eleven months following passage of the law. Eligibility was extended to free white men 21 years and older and orphaned children, each receiving one draw. Married white men with children and widows with children received two draws. Citizens were given at least 10 days notice, by public advertisement, to attend their respective county courthouse in order to register. County justices were responsible for compiling a list of names and number of eligible draws for each registered participant.\footnote{Figure \ref{chronology} details the sequence of the land lottery process.}

\begin{figure}[htbp] 
	\begin{center}
		\begin{chronology}[1]{1802}{1809}{18cm}[22cm]
			\event{\decimaldate{16}{6}{1802}}{Treaty: 16 June 1802}
			\event{\decimaldate{11}{5}{1803}}{1805 law: 11 May 1803}
			\event[\decimaldate{11}{6}{1803}]{\decimaldate{1}{3}{1804}}{1805 registration: 11 May 1803 - 1 March 1804}
			\event[\decimaldate{22}{7}{1805}]{\decimaldate{31}{8}{1805}}{1805 drawing: 22 July 1805 - 31 August 1805}
			\event[\decimaldate{3}{9}{1805}]{\decimaldate{2}{9}{1806}}{1805 grant claiming: 3 September 1805 - 2 September 1806}
			\event{\decimaldate{9}{6}{1806}}{1807 law: 9 June 1806}
			\event[\decimaldate{31}{6}{1806}]{\decimaldate{31}{9}{1806}}{1807 registration: June 1806 - September 1806} 
			%\event[\decimaldate{27}{8}{1806}]{\decimaldate{31}{10}{1806}}{1805 fractions sale: 27 August 1806 - 31 October 1806}
			\event[\decimaldate{10}{8}{1807}]{\decimaldate{24}{9}{1807}}{1807 drawing: 10 August 1807 -24 September 1807}
			\event[\decimaldate{30}{9}{1807}]{\decimaldate{31}{12}{1808}}{1807 grant claiming: 30 September 1807 - December 1808}
			%\event[\decimaldate{1}{11}{1808}]{\decimaldate{31}{12}{1808}}{1807 fractions sale: November 1808 -- December 1808}
			%\event[\decimaldate{10}{1}{1816}]{\decimaldate{1}{1}{1817}}{Reverted lots sale: 10 January 1816 -}
		\end{chronology}
	\end{center}   
		\caption{Timeline of 1805 and 1807 lottery events \citep{graham2010,graham2011}.\label{chronology}}
\end{figure}

The lottery was established under the direction of five managers appointed by the legislature.\footnote{The five managers were Jared Irwin (President), William Barnett, George R. Clayton, Edwin Mounger, and George Watkins. Irwin was a political reformer who, during his first term as Governor of Georgia (1796-1798), signed the bill that nullified the Yazoo Act. Barnett was a state senator from Elbert county. Mounger served as State Treasurer at the time of the lottery (1799-1806). Clayton served as State Treasurer after Mounger (1806-1825). Watkins was a local politician who authored the first \textit{Digest of the Laws of the State of Georgia} with his brother Robert in 1800.} The managers apportioned the blank and prize tickets, which were placed in a wooden ``lottery wheel'' \citep{cadle1991}. The lottery room was arranged with the managers seated at a long table with a volume of four books containing the names of all participants. The managers announced the names and number of registered draws for each participant in alphabetical orde and drew the tickets without replacement as each participant name was announced. For over five weeks, approximately 1,250 tickets were drawn each day (except Sunday) in this manner \citep{graham2004}.

Participants who won a prize, or ``fortunate drawers,'' had 12 months following the drawing to claim their prize. Fortunate drawers were required to pay four dollars per 100 acres in order to obtain the land grant. In some cases, fortunate drawers sold their grants to land speculators, who in turn sold the land to out-of-state settlers \citep{davis1981}. Land speculators often sought out fortunate drawers who drew particularly valuable lots \citep{cadle1991}. The grant records indicate, for instance, 52 grants were obtained jointly by speculators John Forsyth and Lewis Alexander Dugas.\footnote{Forsyth was a Georgia lawyer who launched an illustrious political career in 1808 when he was elected attorney general of Georgia; Forsyth would later serve as the 33rd Governor of Georgia and the 13th U.S. Secretary of State. Dugas was a Georgia businessman. Land speculator Robert Flournoy of Putnam County purchased 36 grants. Flournoy served two consecutive terms as a state senator from Montgomery county starting in 1814.}

\subsection{1807 Georgia Land Lottery}

The Act of 9 June 1806 extended the boundary between Baldwin and Wilkinson counties 45 degrees west to the Ocmulgee River, adding the area north of the line to Baldwin and the area south of the line to Wilkinson \citep{clayton1812}. The surveying process, which was identical to the 1805 lottery, created 38 districts --- 15 to Baldwin and 23 to Wilkinson --- with each square lot of 202.5 acres included in the public lottery. The process of registering participants was similar to the previous lottery, with participants paying 12.5 cents per draw. The eligibility rules were also similar, except that orphan families with both parents deceased received two draws, widows had one draw, free white unmarried females over the age of 21 had one draw, and 1805 fortunate drawers were excluded from participation. 

Unlike the 1805 lottery, there were two lottery wheels used in the 1807 drawing: one containing the names of participants, and the other containing lot numbers. Blank tickets in number equal to difference between the number of registered draws and prize lots were added to the lottery wheel. Tickets from each wheel were drawn simultaneously to form a combined ticket. If the combined ticket included a prize, the winning participant's information was recorded in a grant book. The grant process in the 1807 lottery was similar to that of the 1805 lottery \citep{graham2011}. 
%
%\subsection{1901 Oklahoma Land Lottery} \label{ok-lottery}
%
%At the turn of the 20th Century, the central and western parts of Oklahoma were opened to settlement by non-native Americans. Most of the unassigned lands in central Oklahoma, ceded to the U.S. by the Creek and Seminole Indians following the Civil War, were opened by land run beginning in 1889. In contrast, unassigned land formerly occupied by the Kiowa, Comanche, Apache, and Wichita tribes were opened for settlement by lottery in the summer of 1901 and split into two districts: El Reno and Lawton. 
%
%About 170,000 people registered for the drawing for the chance to win one of 13,000 lots of 160 acres in size.  Registration was open to heads of household 21 years old or older who did not already own more than 160 acres of land. Land officials randomly drew 6,500 envelopes containing participants' information from each of two large hoppers --- one for participants who wanted to claim land in Lawton and the other for those who wanted land in El Reno --- and envelopes were numbered as they were drawn \citep{anderson1997}. Results of the drawing were mailed to participants and posted in the newspapers. Lottery winners had the opportunity to ``stake [a] claim in turn" according to the number on the envelope during the span of 60 days \citep{watson1988}.
%
%Lottery participants were drawn in rank order: the person who drew number one was given first choice, the person who drew number two was given second choice, etc. The draw order was therefore an approximate rank ordering of the amount of lottery wealth, in terms of the option value of the land. \citet{bohanon1998costs} provide a lower-bound estimate of \$496 for the value of a 160-acre lot, which is roughly equal to the average farm income at the time of the 1900 Census. Contemporaneous accounts estimate that the most valuable lots adjacent to the city of Lawton were worth between \$20,000 to \$40,000.\footnote{``Our Uncle Samuel's Lottery and Land Drawing in Oklahoma." \emph{Fort Morgan Times}, Volume XVII, Number 51, August 3, 1901. Available at \url{https://www.coloradohistoricnewspapers.org}.} 