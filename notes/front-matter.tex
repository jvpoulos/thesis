Much of their attention is focused on state-building with respect to the centralization of national power. For instance, American political development (APD) scholars trace the expansion of federal bureaucratic capacity through the merit-based federal civil service system installed in the aftermath of the American Civil War \citep{skowronek1982building,bensel1990,carpenter2001}. 

The American state, however, is organized horizontally and authority is often delegated downward to sub-national units of government. As \citet{novak2008myth} writes, ``trying to gauge the power of the American state or the reach of American public policy by looking simply at the national center or the federal bureaucracy is to miss where much of the action is --- on the local and state levels --- on the periphery."

As pointed out by \citet{greene1986peripheries}, the American state started on the periphery and grew by developing strategies for managing a rapidly expanding frontier territory .

This paper also contributes to the comparative politics literature concerned with land reform and state-building (e.g., Albertus 2015; Murtazashvili and Murtazashvili 2016). Land reform refers to policies designed to establish or redefine property institutions to increase land tenure, and includes policies such as land redistribution, land titling, and decentralization of public land. Land reform is an important tool of state-building, which is broadly defined as efforts to strengthen weak nation-states states through political and economic reforms.

%This paper also contributes to the comparative politics literature concerned with land reform and state-building  \citep[e.g.,][]{albertus2015autocracy, murtazashvili2016does}. Land reform refers to policies designed to establish or redefine property institutions to increase land tenure, and includes policies such as land redistribution, land titling, and decentralization of public land. Land reform is an important tool of state-building, which is broadly defined as efforts to strengthen weak nation-states states through political and economic reforms.% \citet{albertus2015autocracy} theorizes that the successful implementation of land reform requires sufficient administrative capacity, low institutional constraints, and a coalitional split between landowners and the ruling class that provides the ruling class with the incentive to pursue land reform.

%This paper contributes to research concerned with state-building, in which it is generally argued that greater economic power of the ruling class reduces investment in state capacity. A competing argument emerges from the results of the paper that early land reform in the American case increased the economic power of elites by enabling speculators, railroad companies, and other corporations to scoop up huge swaths of valuable land and then act as passive rentiers. The coinciding expansion of railroad access across the frontier increased farm land values, and consequently, state capacity. I show that homesteads significantly increase railroad access western frontier counties over a period extending into the early twentieth century.

Which historical processes are responsible for present-day differences in the capacity of state governments? For example, there exists considerable variation in both the amount and revenue sources of state and local government funding for public education: New York spent almost twice the national average per-pupil, primarily using local (54\%) and state (41\%) revenue sources, while Idaho spent about 60\% of the national average from a combination of state (63\%), local (26\%) and federal (11\%) sources.\footnote{Source: 2014 Annual Survey of School System Finances, U.S. Census Bureau. \url{https://www.census.gov/programs-surveys/school-finances.html}.}

%Who are these landed elites in the Southern frontier that speculate with land titles, perverting the spirit of the Homestead Acts. How did these elites advance their interest? 
%This connects to the reference to Mares and Queralt 2015. 
%In that piece, landed elites legislated in favor of income taxes to shift tax burden away from the agricultural sector. 
%How did landed elites advance their interest in the public land states?

%Why HSA/SHA - upfront with timing 

%Articulate a clear theory from which empirical implications are drawn.
%Is this a story of sectoral redistribution, policy capture, rent-seeking by bureaucrats?