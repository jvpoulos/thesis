\chapter{Hypothesis Testing for Chapter \ref{rnns-causal}} \label{eval}

\citet{abadie2010synthetic} propose a randomization inference approach for calculating the exact distribution of placebo effects under the sharp null hypothesis of no effect. \citet{cavallo2013catastrophic} extends the placebo-based testing approach to the case of multiple (placebo) treated units by constructing a distribution of \emph{average} placebo effects under the null hypothesis. \citet{firpo2018synthetic} derive the conditions under which the randomization inference approach is valid from a finite sample perspective and \citet{hahn2017synthetic} analyze the approach from a repeated sampling perspective.

Randomization $p$-values are obtained following these steps:

\begin{enumerate} 
	\item Estimate the observed test static $\boldsymbol{\hat{\upphi}}$ from (\ref{eq:pointwise}). Averaging over the time dimension results in a $\text{T}_\star$-length array of observed average treatment effects. 
	\item Calculate every possible average placebo treated effect $\upmu$ by randomly sampling without replacement which $\text{J}-1$ control units are assumed to be treated. There are $\mathcal{Q} = \sum\limits_{\text{g}=1}^{\text{J}-1} {\text{J} \choose \text{g}}$ possible average placebo effects. Since calculating $\mathcal{Q}$ can be computationally burdensome for relatively high values of $J$, I artificially set $\mathcal{Q} = 10,000$ in cases when $\text{J} > 16$. The result is a matrix of dimension $\mathcal{Q} \times \text{T}_\star$
	\item Sum over the time dimension the number of $\upmu$ that are greater than or equal to $\boldsymbol{\hat{\upphi}}$.  \label{counts}
\end{enumerate}

Each element of the vector obtained from Step \ref{counts} is divided by $\mathcal{Q}$ to estimate a $\text{T}_\star$-length vector of exact two-sided $p$ values, $\hat{p}$. 

\subsection{Randomization confidence intervals}

Under the assumption that treatment has a constant additive effect $\Delta$, I construct an interval estimate for $\Delta$ by inverting the randomization test. Let $\updelta_\Delta$ be the test statistic calculated by subtracting all possible $\upmu$ by $\Delta$. I derive a two-sided randomization confidence interval by collecting all values of $\updelta_\Delta$ that yield $\hat{p}$ values greater than or equal to significance level $\upalpha=0.05$. I find the endpoints of the confidence interval by randomly sampling 500 values of $\Delta$.