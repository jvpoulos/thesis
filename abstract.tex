% (This file is included by thesis.tex; you do not latex it by itself.)

\begin{abstract}

% The text of the abstract goes here.  If you need to use a \section
% command you will need to use \section*, \subsection*, etc. so that
% you don't get any numbering.  You probably won't be using any of
% these commands in the abstract anyway.

This dissertation argues that early land reforms had consequential long-run impacts on the development of the American state. The state-building role of land reform, frequently discussed in the context of comparative political economy, is often overlooked in the study of American political development. I provide evidence that mid-nineteenth century homestead acts had significant long-run impacts that can help explain contemporary differences in state capacity. I find that homestead policy --- or the homestead entries authorized by homestead policy --- had significant and negative effects on the size of state governments lasting a century after passage of the 1862 Homestead Act. This finding implies that in the absence of homestead policy, the average size of frontier state governments would have exceeded the average size of governments in states that were not directly affected by the policy. Potential causal mechanisms include land inequality, which I show to be positively correlated with state capacity, especially at higher levels of inequality. This empirical relationship is consistent with median voter theories of inequality and redistribution. 

The dissertation contributes to a new generation of data-driven machine learning methods for estimating causal impacts of policy interventions on panel data. Machine learning approaches are capable of automatically choosing appropriate predictors without relying on pre-intervention covariates, limiting the `researcher degrees of freedom' that arise from model specification choices. The proposed method based on recurrent neural networks (RNNs) can learn nonconvex combinations of predictors, are specifically structured for sequential data, and outperform regression-based estimators in high-dimensional data settings. The dissertation also introduces a matrix completion method for counterfactual prediction, which is adaptable to settings with staggered treatment adoption and outperforms RNN-based estimators and other regression-based estimators in small-dimensional data settings.
\end{abstract}
