% (This file is included by thesis.tex; you do not latex it by itself.)

\begin{abstract}

% The text of the abstract goes here.  If you need to use a \section
% command you will need to use \section*, \subsection*, etc. so that
% you don't get any numbering.  You probably won't be using any of
% these commands in the abstract anyway.

This dissertation argues that early land reforms had consequential long-run impacts on the development of the American state. The state-building role of land reform, frequently discussed in the context of comparative political economy, is often overlooked in the study of American political development. I provide evidence that mid-nineteenth century homestead acts had significant long-run impacts that can help explain contemporary differences in the size of state governments. I find that homestead policy ---- or the homestead entries authorized by the policies --- had significant and negative effects on the size of state governments lasting a century after passage of the 1862 Homestead Act. This finding implies that in the absence of homestead policy, the average size of frontier state governments would have been higher than eastern states that were not directly affected by homesteads. I also find that homestead policy had no significant long-term effect on state government school spending. The absence of a positive effect on public schooling may be explained by the failure of homestead policy to sufficiently reduce land inequality, which is correlated with the \emph{de facto} power of wealthy landowners to block education reforms. Land speculators and natural resource companies exploited homestead policies in order to appropriate ``free'' public land.

The dissertation makes a novel methodological contribution in proposing data-driven machine learning alternatives to the popular synthetic control method (SCM) for estimating the effect of a policy intervention on an outcome over time. The proposed method based on recurrent neural networks (RNNs) is less susceptible to $p$-hacking because it does not require the researcher to choose predictors or pre-intervention covariates to construct the synthetic control. Moreover, RNNs do not assume a functional form, can learn nonconvex combinations of control units, and are specifically structured to exploit temporal dependencies in sequential data. The RNN-based methods outperform the SCM in high-dimensional data settings when the number of pre-intervention time periods exceeds the number of control units. The RNN-based approach is compared to a matrix completion method, which is adaptable to settings with staggered treatment adoption and performs well on small-dimensional datasets. Both methods can be understood within the framework of modern causal inference.
\end{abstract}
