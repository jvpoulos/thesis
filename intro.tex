\chapter{Introduction}

This dissertation argues that early land reforms had consequential long-run impacts on the development of the American state. The state-building role of land reform, which includes policies to decentralize public land, is frequently discussed in the context of comparative political economy \citep[e.g.,][]{albertus2015autocracy,murtazashvili2016does}. Several scholars of American Political Development (APD) have studied the implications of land policies designed to open up the western frontier on the developmental trajectory of the U.S. \citep[e.g.,][]{bensel1990,frymer2014rush}; however, my dissertation research is the first to quantify how land policies shaped the American political economy. This area of research is important because policies enacted in the early U.S. helped shape the developmental trajectory of the American political economy.

In the existing APD literature, substantial attention is paid to state-building with respect to the centralization of national power.  For instance, APD scholars trace the expansion of federal bureaucratic capacity through the merit-based federal civil service system installed in the aftermath of the American Civil War \citep{skowronek1982building,bensel1990,carpenter2001}. The American state, however, is organized horizontally and authority is often delegated downward to sub-national units of government. As \citet{novak2008myth} writes, ``trying to gauge the power of the American state or the reach of American public policy by looking simply at the national center or the federal bureaucracy is to miss where much of the action is --- on the local and state levels --- on the periphery." How did public land laws shape the development of state governments?

This dissertation also contributes to the comparative politics literature concerned with land reform and state-building  \citep[e.g.,][]{albertus2015autocracy, murtazashvili2016does}. Land reform refers to policies designed to establish or redefine property institutions to increase land tenure, and includes policies such as land redistribution, land titling, and decentralization of public land. The papers included in this dissertation explore the consequences to two different forms of public land decentralization in the 19th century U.S.: land lotteries and homestead policy.

\section{Land Reform, Inequality and State Capacity}

Land reform is an important tool of state-building, which is broadly defined as efforts to strengthen weak nation-states states through political and economic reforms. \citet{albertus2015autocracy} theorizes that the successful implementation of land reform requires sufficient administrative capacity, low institutional constraints, and a coalitional split between landowners and the ruling class that provides the ruling class with the incentive to pursue land reform.

It is generally argued in the state-building literature that greater economic power of the ruling class reduces investment in state capacity. State capacity refers not only to the ability to raise revenue, but also the state’s ability to implement policies such as public education through redistributive spending \citep{besley2010state}. The canonical model of \citet{meltzer1981rational} predicts a positive relationship between inequality and redistribution because greater inequality implies the median voter is poorer than the average voter, which in turn increases demand for redistribution in majority-rule elections. 

The canonical model assumes parity in the political influence of voters. Economic differences in political influence can arise from the comparative advantage of economic elites in solving collective action problems and benefitting from political intervention \citep{acemoglu2008persistence}. Elites have higher education attainment and earnings abilities \citep{bourguignon2000oligarchy}, and their resource advantage and common interests enable them work toward similar policy goals \citep{winters2009}. Models that allow for differences in political influence across economic groups yield a nonlinearity in the relationship between inequality and redistribution. In \possessivecite{benabou2000unequal} model, for instance, the pivotal voter is wealthier than the median and has the power to block redistribution as inequality increases. But when inequality is too high, the poor can impose redistribution on elites through `universal' majority voting \citep{perotti1993political,saint1993education}. Similarly, greater economic power of the ruling class reduces investment in state capacity in \possessivecite{besley2009origins} framework. 

\citet{galor2009inequality} propose a model where wealthy landowners block education reforms because education favors industrial labor productivity and decreases the value in farm rents. The model implies education reforms will not occur where land is abundant and unequally distributed, and thereby delay the transition from an agricultural to an industrial economy. However, land reforms that sufficiently reduce land inequality diminishes the economic incentives of landowners to block education reforms. 

Qualitative evidence in the U.S. South points to the fact that southern governments did not prioritize early education reforms such as universal public schooling because of the inability of rural planters to capture the returns to human capital investment. Eight of the 11 states of the former Confederacy invested less than 50\% of the national average in per-pupil education expenditures as recently as 1940 \citep{wright1986old}. According to \citet{wright1986old}, the South's historically low priority in education investment cannot be explained by its relative income position. 

\subsection{``Hollowing Out'' Tax Institutions}
	
Landed elites might choose an inefficient organization of the state in order to create inefficiencies in tax collection \citep{acemoglu2011emergence} or ``hollow-out'' tax institutions in order to constrain the state's ability to tax in the future \citep{suryanarayan2017hollowing}. Inequality in this context can be thought of as a proxy for the amount of \emph{de facto} political influence elites have to block reforms and limit the capacity of the state \citep{acemoglu2008persistence}.

\citet{mares2015non} argue that higher levels of land inequality increases the likelihood of adopting new taxes in order to give the landed elite the ability to design a tax structure that shifts the tax burden from the agricultural sector to the manufacturing sector. The authors show empirically that countries with high levels of land inequality are more likely to adopt the income tax.  Consistent with the findings of \citet{mares2015non}, I show that land inequality is positively related to state capacity, especially at higher levels of inequality. 

This finding is in contrast to several empirical studies that establish a negative relationship between land inequality and state capacity in the context of taxes, revenues, and public school spending at the county-level in 1890 and 1930. These studies are based on analyses over shorter time periods. \citet{ramcharan2010inequality}, for instance, finds an inverse relationship between land inequality and county-level property tax revenues in 1890. The authors find that the negative relationship is especially large in rural counties, where landownership tends to be more concentrated. \citet{vollrath2013inequality} establish a negative relationship between land inequality and local property tax revenues in 1890 in northern rural counties, but not in southern rural counties. 

\subsection{Frontier Land Monopolization} 

A competing argument emerges from the dissertation that early land reform in the American case increased the economic power of elites by enabling speculators, railroad companies, and other corporations to scoop up huge swaths of valuable land and then act as passive rentiers.  The U.S. frontier experienced a large-scale transformation of hundreds of millions of public frontier land into private property beginning with the passage of the Land Act of 1820, which permitted the direct sale of public land to private individuals of 80 acres or more for at least \$1.25 an acre, and accelerating after the passage of the homestead acts. 

About 150 million acres of public land, or about 7\% of the total area of frontier states, had already been sold by the time of the passage of the Homestead Act of 1862.\footnote{Source: author's estimates using land patent data described in Section~\ref{state-capacity}.} By the turn of the twentieth century, 250 million acres (11\% of total land area) had been sold, while 100 million acres (4\% of total land area) had been claimed by homestead. In the South, about 50 million acres of public land, or about 31\% of the states' total acreage, had already been sold by before the passage of the Southern Homestead Act of 1866. A substantial rise in the number and total acreage, respectively, of homestead entries in the South and West occurred after the 1889 cash-entry restriction.

The large-scale transfer of public land to railroad companies may have accelerated the increase in frontier land inequality. Land near railroad tracks were often high quality and railroad companies sold excess land to speculators \citep{murtazashvili2013political}. Railroad companies were granted nearly 130 million acres of land between 1862 and 1871 by federal and state governments. About 20\% of the public domain was granted to states or sold or granted to railroads or other corporations, which is comparable to the amount of land granted or sold to homesteaders \citep{shanks2005homestead}.
 
\section{What Will the Railroad Bring Us?} 

From 1790 to the 1840s, state governments were the most active level of government in terms of public goods provision \citep{wallis2000american}. After 1842, infrastructure investments at the local level began to outpace state investments, ushering in an era of growth in local government investments built upon a broadening property tax base that lasted until the early 1930s. For example, Indiana, a frontier state, saw the number of taxable acres in the state double from 7 million in 1838 to 14 million in 1844; although, the value of taxable land per acre in the state was more than halved during the same period (\$8.50 to \$3.71 per acre) due the Panic of 1837 and collapsing speculative land bubble \citep{wallis2004sovereign}. 

Frontier states prioritized spending on banking and transportation projects in competition with each other to raise land values and attract more settlers \citep{sylla1998anatomy}. The promise of increasing land values and future tax revenues led frontier states to sharply increase borrowing in the mid-1830s by selling long-term bonds to finance transportation and banking investments. Frontier states in the Northwest (e.g., Illinois, Indiana, and Michigan) borrowed to invest in canals and railroads, while those in the South (e.g, Arkansas, Florida, and Mississippi) borrowed in order to charter state banks. Indiana, for instance, passed the Mammoth Internal Improvement Act in 1836 that added \$10 million in debt spending for transportation projects such as the Wabash and Erie Canal. The state also changed its property tax structure from a flat tax on land to an \emph{ad valorem} tax on all wealth in order to capture the expected increase in land values that would result from the projects \citep{wallis2004sovereign}. The Panic of 1837 decreased land values and put a hole in the state's tax revenue, causing the state to default on its debt and interest obligations.

Because railroads expanded commerce by making it cheaper to trade, railroad access is theoretically expected to increase returns to farm land, and in turn increase the property tax base. \citet{donaldson2016railroads}, for instance, find that average farm values increased substantially as the railroad network expanded from 1870 to 1890, and estimate that the absence of railroads would have decreased farm land values by 60\%. \citet{atack2011impact} attribute two-thirds of the increase in improved farm acreage in Midwestern states to the expansion of railroad access in the decade prior to the Civil War. As evidence that railroad access increases the property tax base through higher land values, \citet{atack2012impact} find school attendance rates increased in counties that gained access to the rail network between 1850 and 1860. 

Speculators and railroad companies worked together to secure state, county, and municipal grants for local railroads. In \possessivecite{gates1939land} study of Indiana counties, for example, speculators fought county governments subsidizing railroads --- despite possibly benefiting from the lines --- and the speculators' land reduced the tax base of the area, so that counties could not sell bonds to finance railroad subsidies. 

\section{Counterfactual History}

The dissertation also makes a methodological contribution in the development and application of machine learning methods for inferring the causal impacts of policy interventions time-series cross-section data. 

%counterfactual history
%sunstein
%Fogel: Railroads and American Economic Growth: Essays in Econometric History
%Hornbeck RR
%Williamson: Late Nineteenth-Century American Development: A General Equilibrium History

\section{Overview}

The dissertation is comprised of the following three papers related to the political economy of the American frontier. 

\subsection{State-Building through Public Land Disposal? An Application of Matrix Completion for Counterfactual Prediction}
This essay provides evidence that mid-nineteenth century homestead acts had significant long-run impacts that can help explain contemporary differences in the capacity of state governments. The paper applies a matrix completion method to predict the counterfactual time-series of frontier state capacity had there been no homestead acts. Causal estimates signify that homestead policies had significant and long-lasting negative impacts on state government expenditure and revenue. These results are similar to difference-in-difference estimates that exploit variation in the timing and intensity of homestead entries aggregated from 1.46 million individual land patent records.

\subsection{Land Lotteries, Long-term Wealth, and Political Selection} 
This essay asks whether personal wealth can cause individuals to select into office. This question is important and relevant because wealthy individuals might select into office in order to use their power to protect vested interests rather than advance the interests of their constituents. While several studies have studied the effect of officeholding on wealth accumulation, research on the extent to which personal wealth affects the probability of officeholding is much more limited.

The paper takes advantage of the random assignment of land in Georgia at the beginning of the nineteenth century. This random assignment of land generates a meaningful ex-ante exogenous shock to personal wealth, which is expected to reduce the opportunity costs of holding office and may make it more important for the wealthy to hold office. I find no evidence in support of the hypotheses that wealth increases the probability of running for office or holding office and argue that these null results are informative because the estimated effects are not practically different than zero. The absence of a treatment effect suggests that observed cross-sectional correlations between wealth and officeholding are likely due to selection effects. 

\subsection{RNN-Based Counterfactual Time-Series Prediction, with an Application to Homestead Acts and State Capacity}
This essay empirically evaluates the RNN-based method for estimating the effect of a policy intervention on an outcome over time. The proposed method offers a more principled approach than the synthetic control method (SCM) because it automatically selects the most relevant predictors at each time-step, without relying on pre-treatment covariates. RNNs are specifically designed to handle sequential data and have two important built-in advantages over SCM: first, RNNs are capable of handling multiple treated units, which is useful because the networks can share parameters across treated units and can thus generate more precise predictions in settings where treated units share similar data-generating processes; second, RNNs can learn nonconvex combinations of predictors, which is beneficial when the data-generating process underlying the outcome of interest depends nonlinearly on the history of predictors. 

In an empirical application, I estimate the causal impacts of homestead acts on state government education spending.