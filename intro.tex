\chapter{Introduction} \label{diss-intro}

This dissertation argues that early land reforms had consequential long-run impacts on the development of the American state. The state-building role of land reform, which includes policies to decentralize public land, is frequently discussed in the context of comparative political economy \citep[e.g.,][]{albertus2015autocracy, murtazashvili2016does}. Land reform refers to policies designed to establish or redefine property institutions to increase land tenure, and includes policies such as land redistribution, land titling, and decentralization of public land. 

Substantial attention in the American Political Development (APD) literature is paid to state-building with respect to the centralization of national power. For instance, APD scholars trace the expansion of federal bureaucratic capacity through the merit-based federal civil service system installed in the aftermath of the American Civil War \citep{skowronek1982building,bensel1990,carpenter2001}. The American state, however, is organized horizontally and authority is often delegated downward to sub-national units of government. As \citet{novak2008myth} writes, ``trying to gauge the power of the American state or the reach of American public policy by looking simply at the national center or the federal bureaucracy is to miss where much of the action is --- on the local and state levels." This dissertation explores the development of sub-national governments across time, and specifically focuses on the role of mid-19th century public land laws in shaping the capacity of state governments.

The American state started on the periphery and grew by developing strategies for managing a rapidly expanding frontier territory \citep{greene1986peripheries}. Several scholars theorize a relationship between land policy and the development of the national government by means of (1) improving bureaucratic capacity to administer land, and (2) outsourcing the defense of contested territories to settlers \citep{frymer2014rush,frymer2017building}. These studies do not consider the long-term implications of land policies on the development of state capacity at the sub-national level.

This dissertation is the first to quantify how land policies shaped the developmental trajectory of the American political economy. It explores the consequences to two different mechanisms of public land decentralization in the 19th century U.S.: land lotteries and homestead policies. The latter offered a free land grant to citizens willing to build on an improve a plot of land of predetermined size for a number of years; land lotteries involved the random allocation of the right to file a land grant, without a homesteading requirement. Both mechanisms were designed to rapidly populate the frontier with mostly white settlers, thereby reducing the costs of defending unsecured territory for otherwise resource-constrained governments.

\section{Land reform, inequality and state capacity}

Land reform is an important tool of state-building, which is broadly defined as efforts designed to strengthen weak nation-states states through political and economic reforms. It is generally argued in the state-building literature that greater economic power of the ruling class reduces investment in state capacity. State capacity refers not only to the ability to raise revenue, but also the state’s ability to implement policies such as public education through redistributive spending \citep{besley2010state}. The canonical model of \citet{meltzer1981rational} predicts a positive relationship between inequality and redistribution because greater inequality implies the median voter is poorer than the average voter, which in turn increases demand for redistribution in majority-rule elections. 

The canonical model assumes parity in the political influence of voters. However, economic differences in political influence can arise from the comparative advantage of economic elites in solving collective action problems and benefitting from political intervention \citep{acemoglu2008persistence}. Elites have higher education attainment and earnings abilities \citep{bourguignon2000oligarchy}, and their resource advantage and common interests enable them work toward similar policy goals \citep{winters2009}. 

Models that allow for differences in political influence across economic groups yield a nonlinearity in the relationship between inequality and redistribution. In \possessivecite{benabou2000unequal} model, for instance, the pivotal voter is wealthier than the median and has the power to block redistribution as inequality increases. But when inequality is too high, the poor can impose redistribution on elites through `universal' majority voting \citep{perotti1993political,saint1993education}. In \possessivecite{besley2009origins} framework, greater economic power of the ruling class reduces investment in state capacity. 

\citet{galor2009inequality} propose a model where wealthy landowners block education reforms because education favors industrial labor productivity at the expense of the agricultural sector. Their model implies education reforms will not occur where land is abundant and unequally distributed, and thereby delays the transition from an agricultural to an industrial economy. However, land reforms that sufficiently reduce land inequality diminishes the economic incentives of landowners to block education reforms. 

\subsection{``Hollowing out'' or sectoral shift of new taxes?}
	
Landed elites might choose an inefficient organization of the state in order to create inefficiencies in tax collection \citep{acemoglu2011emergence} or ``hollow-out'' tax institutions in order to constrain the state's ability to tax in the future \citep{suryanarayan2017hollowing}. Inequality in this context can be thought of as a proxy for the amount of \emph{de facto} political influence elites have to block reforms and limit the capacity of the state \citep{acemoglu2008persistence}.

\citet{mares2015non} argue that higher levels of land inequality increases the likelihood of adopting new taxes in order to give the landed elite the ability to design a tax structure that shifts the tax burden from the agricultural sector to the manufacturing sector. The authors show empirically that countries with high levels of land inequality are more likely to adopt the income tax. 

Consistent with the findings of \citet{mares2015non}, I show that land inequality is positively related to state capacity, especially at higher levels of inequality. This empirical relationship is based on a bivariate regression of per-capita revenue or expenditure on lagged land inequality over the period of 1860 to 1950 (Figure \ref{fig:ineq-capacity}): states with high levels of inequality in the previous period tend to have higher state capacity in the current period.\footnote{The lagged inequality measure helps control for reverse causality in the form of state policies shaping the distribution of landownership.} I show in Figure \ref{fig:ineq-taxpc} that land inequality is also positively related to tax capacity --- the bureaucratic ability of governments to raise taxes from multiple sources in order to finance policies --- in public land state counties during the period 1870 to 1950.\footnote{Fiscal capacity is measured by the total taxes collected by counties (Tax1) --- available for the years 1870, 1880, 1922, 1932, and 1942--- and total taxes collected by all local governments within counties (Tax2), available for 1870, 1880, and 1932. Both measures are derived from census tax data recorded at the county-level \citep{rhode2003assessing,haines2010}.}

These empirical findings are in contrast to several empirical studies that establish a negative relationship between land inequality and tax- or state capacity. These studies are based on analyses over shorter time periods. \citet{ramcharan2010inequality}, for instance, finds an inverse relationship between land inequality and county-level property tax revenues in 1890. The authors find that the negative relationship is especially large in rural counties, where landownership tends to be more concentrated. \citet{vollrath2013inequality} establish a negative relationship between land inequality and local property tax revenues in 1890 in northern rural counties. 

\subsection{Culture and migration to the frontier}

It is commonly believed in the political economy literature that state capacity in general is easier to raise in more ethnically or culturally homogeneous societies \citep{besley2010state}. It has also long been viewed in by sociologists \citep{meyer1979public} and more recently, political economists \citep{alesina2013nation,bandiera2018nation}, that the rise of public schooling in the U.S. during the 19th century can be explained as a nation-building policy enacted primarily by state governments, which were the most active level of government in terms of public goods provision from 1790 to the 1840s \citep{wallis2000american}. \citet{bandiera2018nation}, for instance, argues that states adopted compulsory schooling laws as a nation-building tool to instill civic values to the millions of foreign migrants who arrived to the country between 1850 and 1914, during the `Age of Mass Migration.' 

During this time period, the frontier attracted foreign migrants along with workers from eastern U.S. cities, many of whom were the lower end of the skill distribution. \citet{bazzi2017frontier} find evidence that frontier settlers were disproportionately illiterate and foreign-born. The negative selection of westward migrants during this period is consistent with theory of the frontier as a `safety valve' for relieving congested urban labor markets in eastern states \citep{turner1956significance, ferrie1997migration}.

Beside the negative selection in terms of labor market skill, \citet{bazzi2017frontier} argue that the frontier attracted individualists who would be able to thrive in the harsh conditions characterizing the frontier experience; e.g., the social isolation due to low population density on the frontier. The authors find that counties with longer historical frontier experience exhibit lower contemporary property tax rates, and their citizens are more opposed to taxation and redistribution. 

\section{Land monopolization and rent-seeking on the frontier} 

The U.S. frontier experienced a large-scale transformation of hundreds of millions of public frontier land into private property beginning with the passage of the Land Act of 1820, which permitted the direct sale of public land to private individuals of 80 acres or more for at least \$1.25 an acre, and accelerating after the passage of the homestead acts. About 150 million acres of public land, or about 7\% of the total area of frontier states, had already been sold by the time of the passage of the Homestead Act (HSA) of 1862, which opened for settlement hundreds of millions of acres of western frontier land.\footnote{Source: author's estimates using land patent data described in Chapter \ref{land-reform}. The cumulative number of total acres disbursed by cash entry is plotted in Figure \ref{sales-acres-time}.} By the turn of the twentieth century, 250 million acres (11\% of total land area) had been sold, while 100 million acres (4\% of total land area) had been claimed by homestead. In the South, about 50 million acres of public land, or about 31\% of the states' total acreage, had already been sold by before the passage of the Southern Homestead Act (SHA) of 1866. A substantial rise in the total acreage authorized under the HSA occurred after the 1889 cash-entry restriction imposed by the U.S. Congress (Figure \ref{homesteads-acres-time}).

In theory, homesteading laws were designed to offer greater economic opportunities to yeoman farmers, many of whom were migrants from eastern states where market capitalism had already developed \citep{kulikoff1992agrarian}. In practice, public land laws were exploited by land speculators, ranchers, miners, and loggers, to accumulate public land and extract natural resources during the early stages of capitalist development \citep{murtazashvili2013political}. In Chapter \ref{land-reform}, I find that homesteads significantly decrease future land inequality in frontier states over a period extending to the mid-20th century; although, the magnitude of the estimated effect is negligible.

The failure of the homestead acts in promoting a more egalitarian pattern of land ownership has important implications for the development of the frontier state. Figure \ref{aland-gini-state-time} shows the evolution of land inequality in western public land states is nearly identical to the course of inequality in state land states; i.e., states not crafted from federal public lands and unaffected by homestead policy. Land inequality in southern public land states declined following the passage of the SHA, but southern state land states also declined at a similar rate. 

Homestead policy may have failed to create a more equitable land distribution in part due to the accumulation of public land by speculators and corporations through corrupt practices, such as the use of dummy entrymen, which is the practice of paying individuals to stake out a homestead in order to extract resources from the land with no intention of filing for the final patent. In the South, dummy entrymen were used by timber and mining companies to extract resources while the cash entry restriction of the SHA was in effect. When the restriction was removed, there was no need for fraudulent filings because the larger companies could buy land in unlimited amounts at a nominal price until public land were withdrawn from cash entry in 1889 \citep{gates1940federal,gates1979federal}. The same pattern of fraudulent filings existed in the West, where \citet{murtazashvili2013political} argues that speculators benefited disproportionately from land laws because the economic balance of power tilted toward the wealthy. \citet{gates1942role} characterizes western speculators who bought land in bulk prior to the 1889 restriction as being influential in state and local governments, resistant to paying taxes, and opposed to expenditures except for transportation facilities close to their land.

\section{Railroad access as a potential confound} 

In Chapter \ref{land-reform}, a measure of railroad access based on the total miles of operational track per square mile is included in the state capacity models in order to control for sample selection bias arising from differences in settler access to frontier lands. The idea is that states that had more connections to the rapidly expanding railroad network (Figure \ref{rr-map}) are more likely be exposed to homesteads. 

Railroad access can also be thought of as a potential confounding variable, affecting both the probability of exposure to homesteads and future state capacity. Railroads expanded commerce by making it cheaper to trade, and railroad access is theoretically expected to increase returns to farm land, and in turn increase the property tax base. \citet{donaldson2016railroads}, for instance, find that average farm values increased substantially as the railroad network expanded from 1870 to 1890, and estimate that the absence of railroads would have decreased farm land values by 60\%. \citet{atack2011impact} attribute two-thirds of the increase in improved farm acreage in Midwestern states to the expansion of railroad access in the decade prior to the Civil War. As evidence that railroad access increases the property tax base through higher land values, \citet{atack2012impact} find school attendance rates increased in counties that gained access to the rail network between 1850 and 1860.

Frontier states prioritized spending on transportation projects in competition with each other to raise land values and attract more settlers \citep{sylla1998anatomy}. The promise of increasing future tax revenues led frontier states to sharply increase borrowing in the mid-1830s by selling long-term bonds to finance transportation investments. Frontier states in the Northwest borrowed to invest in canals and railroads. For example, Indiana passed the Mammoth Internal Improvement Act in 1836 that added \$10 million in debt spending for transportation projects such as the Wabash and Erie Canal. The state also changed its property tax structure from a flat tax on land to an \emph{ad valorem} tax on all wealth in order to capture the expected increase in land values that would result from the projects \citep{wallis2004sovereign}.\footnote{Indiana's plan backfired when the Panic of 1837 decreased land values and put a hole in the state's tax revenue, causing the state to default on its debt and interest obligations.}

Speculators and railroad companies worked together to secure state, county, and municipal grants for local railroads \citep{gates1939land}. Land near railroad tracks were often high quality and railroad companies sold excess land to speculators \citep{murtazashvili2013political}. Railroad companies were granted nearly 130 million acres of land between 1862 and 1871 by federal and state governments. About 20\% of the public domain was granted to states or sold or granted to railroads or other corporations, which is comparable to the amount of land granted or sold to homesteaders \citep{shanks2005homestead}. 

\section{Summary of causal mechanisms}

Figure \ref{fig:causal-graph} summarizes the causal mechanisms underlying the relationship between homesteads and state capacity in a directed graph. Homestead policy (or the homestead entries authorized by the policy) is expected to decrease land inequality by the process of land decentralization of public land in grants of 160 acres.

The theoretical direction of the impact of land inequality on state capacity is ambiguous. Most prevailing models of political economy, which posit an inverse relationship between inequality and state capacity. Empirically, I find that land inequality is positively correlated with state capacity. This empirical relationship is consistent with \cite{mares2015non}: higher levels of land inequality increases the likelihood of governments adopting new taxes in order to give the landed elite the ability to design a favorable tax structure. When estimating the causal impact of land inequality on state capacity, reverse causality is a problem for identification: governments' ability to implement redistributive policies can also shape the distribution of landownership.

In competition with each other to raise land values and attract more settlers, frontier state governments financed railroad projects and granted land to railroad companies in order to increase railroad access to the state. Previous empirical work suggests that the relationship between railroads and state capacity goes both ways: railroad access increases the property tax base through higher land values. 

Lastly, railroad access potentially confounds the relationship between homesteads and state capacity because the expansion of the railroad network increases settlers' access to frontier lands. 

\section{Overview}

The dissertation is comprised of the following three essays on the political economy of the American frontier. Chapters \ref{land-lotteries} and \ref{counterfactual-history} serve as transitional sections. Chapter \ref{diss-conclusion} summarizes the key empirical findings of the dissertation and their implications and discusses possible directions for future research related to the political economy of the American frontier and counterfactual prediction.

\paragraph{Chapter \ref{land-reform}: State-Building through Public Land Disposal? An Application of Matrix Completion for Counterfactual Prediction}
The essay applies a matrix completion method to predict the counterfactual time-series of frontier state capacity had there been no homestead acts. Causal estimates signify that homestead policies had significant and long-lasting negative impacts on state government expenditure and revenue that lasted a century following their implementation. These results are similar to difference-in-difference (DID) estimates that exploit variation in the timing and intensity of homestead entries aggregated from 1.46 million individual land patent records.

The direction of these estimates align with the view that homestead policies were exploited by land speculators and natural resource companies and that the rents from public land were appropriated by the private sector. I explore land inequality as a possible causal mechanism underlying the relationship between land reform and state capacity, and provide evidence of a positive relationship between land inequality and state government finances and that the slope of correlation increases at higher levels of inequality. I also present DID estimates that reveal per-capita homesteads significantly lowered land inequality in frontier states.

This essay makes a methodological contribution in applying matrix completion for estimating causal impacts of policy interventions on panel data. The method can be easily understood within the framework of modern causal inference, is adaptable to settings with staggered treatment adoption, and outperforms several other regression-based estimators in a battery of placebo tests.

\paragraph{Chapter \ref{ga-lottery}: Land Lotteries, Long-term Wealth, and Political Selection} 
This essay asks whether personal wealth can cause individuals to select into office. This question is important and relevant because wealthy individuals might select into office in order to use their power to protect vested interests rather than advance the interests of their constituents. While several studies have studied the effect of officeholding on wealth accumulation, research on the extent to which personal wealth affects the probability of officeholding is much more limited.

The essay takes advantage of the random assignment of land in Georgia at the beginning of the nineteenth century. This random assignment of land generates a meaningful ex-ante exogenous shock to personal wealth, which is expected to reduce the opportunity costs of holding office and may make it more important for the wealthy to hold office. I find no evidence in support of the hypotheses that wealth increases the probability of running for office or holding office and argue that these null results are informative because the estimated effects are not practically different than zero. The absence of a treatment effect suggests that observed cross-sectional correlations between wealth and officeholding are likely due to selection effects. 

\paragraph{Chapter \ref{rnns-causal}: RNN-Based Counterfactual Prediction, with an Application to Homestead Policy and Public Schooling}

This essay makes a methodological contribution in proposing a novel alternative to the SCM for estimating the effect of a policy intervention on an outcome over time. The proposed method based on recurrent neural networks (RNNs) is less susceptible to $p$-hacking because it does not require the researcher to choose predictors or pre-intervention covariates to construct the synthetic control. Moreover, RNNs do not assume a functional form, can learn nonconvex combinations of control units, and are specifically structured to exploit temporal dependencies in sequential data.

The RNN-based estimators require sufficient pre-period observations in order to learn an informative representation of the control units, and consequently perform comparatively worse than the SCM on small-dimensional datasets such as those featured in the original synthetic control papers. The RNN-based methods outperform the SCM in high-dimensional data settings when the number of pre-intervention time periods exceeds the number of control units.

In the empirical application, I estimate the causal impacts of the HSA on state government education spending. I find that homestead policy had positive long-run impacts on public education spending, although the impacts are not statistically significant when averaging across the entire post-intervention period. Time-specific causal estimates suggest that the HSA had positive and significant impacts on state government education spending fifty years after the first homestead entry in 1869. The estimated increase in education spending attributable to homestead policy translates to about 3\% of the total school expenditures per-capita in 1929.