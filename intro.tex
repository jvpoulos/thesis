\chapter{Introduction}

This dissertation argues that early land reforms had consequential long-run impacts on the development of the American state. The state-building role of land reform, which includes policies to decentralize public land, is frequently discussed in the context of comparative political economy \citep[e.g.,][]{albertus2015autocracy,murtazashvili2016does}. Several scholars of American Political Development (APD) have studied the implications of land policies designed to open up the western frontier on the developmental trajectory of the U.S. \citep[e.g.,][]{bensel1990,frymer2014rush}; however, my dissertation research is the first to quantify how land policies shaped the American political economy. 

In the existing APD literature, substantial attention is paid to state-building with respect to the centralization of national power.  For instance, APD scholars trace the expansion of federal bureaucratic capacity through the merit-based federal civil service system installed in the aftermath of the American Civil War \citep{skowronek1982building,bensel1990,carpenter2001}. The American state, however, is organized horizontally and authority is often delegated downward to sub-national units of government. As \citet{novak2008myth} writes, ``trying to gauge the power of the American state or the reach of American public policy by looking simply at the national center or the federal bureaucracy is to miss where much of the action is --- on the local and state levels --- on the periphery." 

How did public land laws shape the development of state governments? This area of research is important because policies enacted in the early U.S. helped shape the developmental trajectory of the American political economy. %More on contemporary relevance: How homesteads shaped present political economy

This paper also contributes to the comparative politics literature concerned with land reform and state-building  \citep[e.g.,][]{albertus2015autocracy, murtazashvili2016does}. Land reform refers to policies designed to establish or redefine property institutions to increase land tenure, and includes policies such as land redistribution, land titling, and decentralization of public land. Land reform is an important tool of state-building, which is broadly defined as efforts to strengthen weak nation-states states through political and economic reforms. \citet{albertus2015autocracy} theorizes that the successful implementation of land reform requires sufficient administrative capacity, low institutional constraints, and a coalitional split between landowners and the ruling class that provides the ruling class with the incentive to pursue land reform.

It is generally argued in the state-building literature that greater economic power of the ruling class reduces investment in state capacity. A competing argument emerges from the results of the paper that early land reform in the American case increased the economic power of elites by enabling speculators, railroad companies, and other corporations to scoop up huge swaths of valuable land and then act as passive rentiers.

The dissertation also makes a methodological contribution in the development and application of machine learning methods for inferring the causal impacts of policy interventions time-series cross-section data. 

\paragraph*{State-Building through Public Land Disposal? An Application of Matrix Completion for Counterfactual Prediction}
This essay provides evidence that mid-nineteenth century homestead acts had significant long-run impacts that can help explain contemporary differences in the capacity of state governments. The paper applies a matrix completion method to predict the counterfactual time-series of frontier state capacity had there been no homestead acts. Causal estimates signify that homestead policies had significant and long-lasting negative impacts on state government expenditure and revenue. These results are similar to difference-in-difference estimates that exploit variation in the timing and intensity of homestead entries aggregated from 1.46 million individual land patent records.

\paragraph*{RNN-Based Counterfactual Time-Series Prediction, with an Application to Homestead Acts and State Capacity}
This essay empirically evaluates the RNN-based method for estimating the effect of a policy intervention on an outcome over time. The proposed method offers a more principled approach than the synthetic control method (SCM) because it automatically selects the most relevant predictors at each time-step, without relying on pretreatment covariates. RNNs are specifically designed to handle sequential data and have two important built-in advantages over SCM: first, RNNs are capable of handling multiple treated units, which is useful because the networks can share parameters across treated units and can thus generate more precise predictions in settings where treated units share similar data-generating processes; second, RNNs can learn nonconvex combinations of predictors, which is beneficial when the data-generating process underlying the outcome of interest depends nonlinearly on the history of predictors.

In an empirical application, I extend data from a field experiment that investigates the effect of randomized radio ads on electoral competition to an observational time-series setting. I find that RNNs outperform SCM in recovering the ground-truth experimental estimate. I also run placebo tests on three datasets introduced by the SCM literature and find that RNNs achieve lower error rates than SCM, while maintaining comparable false positive rates.

%Which historical processes are responsible for present-day differences in the capacity of state governments? For example, there exists considerable variation in both the amount and revenue sources of state and local government funding for public education: New York spent almost twice the national average per-pupil, primarily using local (54\%) and state (41\%) revenue sources, while Idaho spent about 60\% of the national average from a combination of state (63\%), local (26\%) and federal (11\%) sources.\footnote{Source: 2014 Annual Survey of School System Finances, U.S. Census Bureau. \url{https://www.census.gov/programs-surveys/school-finances.html}.}

\paragraph*{Land Lotteries, Long-term Wealth, and Political Selection} 
This essay asks whether personal wealth can cause individuals to select into office. This question is important and relevant because wealthy individuals might select into office in order to use their power to protect vested interests rather than advance the interests of their constituents. While several studies have studied the effect of officeholding on wealth accumulation, research on the extent to which personal wealth affects the probability of officeholding is much more limited.

The paper takes advantage of the random assignment of land in Georgia at the beginning of the nineteenth century. This random assignment of land generates a meaningful ex-ante exogenous shock to personal wealth, which is expected to reduce the opportunity costs of holding office and may make it more important for the wealthy to hold office. I find no evidence in support of the hypotheses that wealth increases the probability of running for office or holding office and argue that these null results are informative because the estimated effects are not practically different than zero. The absence of a treatment effect suggests that observed cross-sectional correlations between wealth and officeholding are likely due to selection effects. 