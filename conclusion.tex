\chapter{Conclusion} \label{diss-conclusion}

This dissertation argues that early land reforms had consequential long-run impacts on the development of the American state. The state-building role of land reform is frequently discussed in the context of comparative political economy, but rarely in the context of American political economy. I provide evidence that mid-nineteenth century homestead acts had significant long-run impacts that can help explain contemporary differences in the capacity of state governments. The key empirical findings of the dissertation and their implications are summarized as follows.

\begin{description}
	\item[State capacity] Homestead policy, or the homestead entries authorized by the policies, had a significant and negative effect on the size of stage governments as measured by per-capita revenue and expenditure over the period of 1869 to 1982. This finding implies that in the absence of homestead policy, the average capacity of frontier state governments would have been higher than eastern states that were not directly affected by homestead policy. The direction of the point estimates are in line with the view of \citet{gates1968history} and \citet{murtazashvili2013political} that homestead policies were exploited by land speculators and natural resource companies. 
	\item[Land inequality] Homestead policies are expected to lower land inequality by fixing land grants to 160 acres, thereby encouraging farm sizes to approach their ideal scale. I find that homestead entries significantly decreased land inequality in frontier states during the period of 1870 to 1950. Theoretically, we expect land reforms that sufficiently reduce inequality to diminish the incentives of wealthy landowners to block investments in state capacity \citep{besley2009origins,galor2009inequality}; however, the relatively small magnitude of the estimate implies that the reduction in inequality may not have been sufficient enough to diminish the \emph{de facto} power of the wealthy landowners. 
	
	I also show that land inequality and state capacity are positively related during the same period. These pair of findings demonstrate the role of land inequality as a causal mechanism behind the previously described negative relationship between homestead policy and state capacity. 
	
	\item[Public education] I find that homestead policy had no significant overall effect on public education spending at the state-level during the period of 1869 to 1942. Fifty years following the first homestead entry, an estimated increase in public school spending attributable to homestead policy is equivalent to about 3\% of total annual per-capita public education spending. The absence of a significant effect over the entire post-intervention period may be explained by the failure of homestead policy to sufficiently reduce land inequality. 
	
	\item[Political selection] Using the first two Georgia land lotteries as a natural experiment, I examine the individual-level impacts of public land decentralization by lottery. The random assignment of land generates an exogenous shock to personal wealth, which is expected to reduce the opportunity costs of holding office. I find no effect of wealth on running for office or holding office. The absence of a treatment effect implies that observed  correlations between wealth and officeholding are likely due to selection effects.
	
\end{description}

In the remaining sections, I discuss possible directions for future research related to the political economy of the American frontier and counterfactual prediction. 

\section{Future research on the American frontier}

Three directions for future research related to the political economy of the American frontier are worth attention. 

\subsection{Environmental externalities} One direction is related to the work of \citet{hansen2001us} on the harmful consequences of homesteads on the western frontier. The authors argue that the 160-acre homesteads contributed to the Dust Bowl of the 1930s because they were too small to be viable and were more likely to fail during drought. What were the long-run environmental externalities of homesteads on the frontier? One can aggregate the millions of individual homestead entries described in Chapter \ref{land-reform} to the share of settled land, and then use the measure to estimate the consequences of collective settler decisions on drought, soil erosion, and farm failure. 

\subsection{Suffrage} The second direction is related to a question posed by \citet{engerman2005evolution}: ``Why were frontier states more liberal in extending the franchise than the original states that had long been settled?'' The authors argue that the observed pattern of frontier states having more liberal suffrage provisions is due to the conditions of the frontier states rather than the preferences of the national government -- namely, the labor-scarce frontier states had an economic incentive to freely extend suffrage in order to attract eastern migrants and foreign migrants, many of whom were accustomed to suffrage rights in their native countries. One can estimate the impact of per-capita homestead entries on the variation over state and time in the proportion of adult white males who voted, which is available in \citet{mccormick1960new}. As a related inquiry, one might examine how homestead policies impacted the extension of women's suffrage in western states \citep{teele2018west}, and how the expansion of women's suffrage consequently grew the size of western state governments \citep{lott1999did}.

\subsection{Railroads} Recent attention has been paid to the railroad expansion in the 19th century and increasing returns to farm land  \citep{donaldson2016railroads}. After 1842, infrastructure investments at the local level began to outpace state investments, ushering in an era of growth in local government investments built upon a broadening property tax base that lasted until the early 1930s. How did railroad expansion shape the development of tax capacity of local governments? County-level data on railroad access, described in Chapter \ref{land-reform}, can be merged with information on taxes collected by counties during 1870 to 1942 and taxes collected by all local governments within the county during 1870 to 1992 \citep{rhode2003assessing}. In the early 1850s, the U.S. Secretary of War sent expeditions to select routes for a possible railroad from the Mississippi River to the Pacific Ocean \citep{blake1857geological,baird1858reports}, and these routes were later abandoned by Congress as it was deemed there was no commercial reason for a transcontinental railroad. The surveyed routes offer the possibility of a falsification test in the style of \citep{donaldson2018railroads} for testing the null hypothesis that railroad access resulting from the surveyed routes had an impact on the tax capacity of intersecting county governments.\footnote{Maps of the surveyed railroad routes are available in the Library of Congress Geography and Map Division Washington: \url{https://www.loc.gov/item/gm70005008/}.}

\section{Future research on counterfactual prediction}

On the methodological side, further exploration of the domain adaptation problem as it relates to counterfactual prediction is warranted. 

Propensity score reweighting of the training loss function is the standard correction technique for data settings with sample selection bias \citep{cortes2008sample}. In Chapter \ref{rnns-causal}, propensity score weights are based on the probability of receiving treatment, conditional on pre-treatment covariates, in finite data sets. More work can be done on how to measure the extent to which the propensity score estimation error can affect counterfactual predictions. 

In the high-dimensional placebo tests in Chapter \ref{rnns-causal}, the RVAE achieves comparable accuracy despite being self-supervised; i.e., learning representations of the inputs without outputs. Future research might examine how to modify the RVAE to achieve balanced representations; i.e., representations that minimize the discrepancy between the distributions of treated and control groups, similar to the work of \citet{johansson2016learning}.


