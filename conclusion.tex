\chapter{Conclusion}

My dissertation argues that early land reforms had consequential long-run impacts on the development of the American state. The state-building role of land reform is frequently discussed in the context of comparative political economy, but rarely in the context of American political economy. I provide evidence that mid-nineteenth century homestead acts had positive long-run impacts that can help explain contemporary differences in the capacity of state governments.

% Two avenenues for future reserach: homesteads/environmental externalities, Raildroads/fiscal capacity
My proposed research is to investigate the long-run environmental externalities of land reforms on the American frontier. Specifically, I will aggregate millions of individual settler decisions to the share of settled land and estimate the consequences of collective settler decisions on drought, soil erosion, and farm failure. This research will be informative for research in comparative political economy on the long-run environmental impacts of public land reform.

%hansen2001us

%*Railroads <-> fiscal capacity
%measure fiscal capacity (pc taxes) in the absence of railroad access
%railroad construction optimal or driven by rent-seeking?
%
%State Capacity and Economic Development: A Network Approach
%Railroad expansion -> tax capacity, public education
%
%Pacific RR Act of 1862 (can get grants from GLO)
%12 Stat. 489
%Continuous DD
%Counterfactuals: potential Pacific RR routes
%Donaldson India RR paper
%
%Compare with RR access
%(optimal RR development) 