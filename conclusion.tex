\chapter{Conclusion}

My dissertation argues that early land reforms had consequential long-run impacts on the development of the American state. The state-building role of land reform is frequently discussed in the context of comparative political economy, but rarely in the context of American political economy. I provide evidence that mid-nineteenth century homestead acts had positive long-run impacts that can help explain contemporary differences in the capacity of state governments.

% Two avenenues for future reserach: homesteads/environmental externalities, Raildroads/fiscal capacity
My proposed research is to investigate the long-run environmental externalities of land reforms on the American frontier. Specifically, I will aggregate millions of individual settler decisions to the share of settled land and estimate the consequences of collective settler decisions on drought, soil erosion, and farm failure. This research will be informative for research in comparative political economy on the long-run environmental impacts of public land reform.

%Measure of suffrage/ democratic participation (Engerman 2005)
%Why were frontier states more liberal in extending the than the original states that had long been settled? (pg. 889/ ftn. 14)
%See McCormick, "New Perspectives" and Second American Party System, for discussion of the variation over state and time in the proportion of adult white males who voted
%If the right to participate in the political process was desirable to potential migrants, the new states thus had an economic incentive to adopt liberal suffrage provisions
%Women's suffrange (ftn. 14)
%the pattern of frontier areas or new states choosing to extend the franchise more broadly than their neighboring states to the East does not appear to have been driven by the preferences of the U.S. Congress, but rather by conditions in those states 
%
%Scarcity of labor is a condition characteristic of frontier areas, as well perhaps of newly settled regions with small native populations, that might have encouraged new states to place fewer restrictions on who had the right to vote
%
%
%2018. Teele, Dawn. “How the West Was Won: Competition, Mobilization, and Women's Enfranchisement in the United States.”   Journal of Politics, 80(2): 442-461.
%
%Lott, John R. Jr., and Lawrence W. Kenny. "Did Women's Suffrage Change the Size and Scope of Government?" Journal of Political Economy 10 (December 1999): 1163-98.
%
%hansen2001us
%work by Libecap and Hansen on the harmful consequences of homesteads in the Far West
%Property Rights to Frontier Land and Minerals: US Exceptionalism - Libecap 2018

%*Railroads <-> fiscal capacity
%measure fiscal capacity (pc taxes) in the absence of railroad access
%railroad construction optimal or driven by rent-seeking?
%
%State Capacity and Economic Development: A Network Approach
%Railroad expansion -> tax capacity, public education
%
%Pacific RR Act of 1862 (can get grants from GLO)
%12 Stat. 489
%Continuous DD
%Counterfactuals: potential Pacific RR routes
%Donaldson India RR paper
%
%Compare with RR access
%(optimal RR development) 